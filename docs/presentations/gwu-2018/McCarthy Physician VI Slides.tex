\documentclass[t]{beamer}
\usetheme[progressbar=frametitle]{metropolis}
\usefonttheme{professionalfonts}
\usepackage{appendixnumberbeamer}

\usepackage{booktabs}
\usepackage[scale=2]{ccicons}

\usepackage{graphics,graphicx,amssymb,amsmath,pgf,comment,hyperref}
%\usepackage[xcolor=pst]{pstricks}
\usepackage{array}
\usepackage{pgfshade}
\usepackage[round]{natbib}
\usepackage[absolute,overlay]{textpos}
\usepackage{pifont}
\usepackage{dcolumn}
\usepackage{textpos}
\usepackage{color}					
\usepackage{xcolor,colortbl}
\usepackage{tikz}
\usepackage{bbm}
\usepackage{curves}
\usepackage{mathtools}
\usepackage{times}
\usepackage{verbatim}
\usetikzlibrary{snakes,arrows,shapes,positioning}
\def\augie{\fontencoding{T1}\fontfamily{augie}\selectfont}

\usepackage{pgfplots}
\usepgfplotslibrary{dateplot}

\usepackage{xspace}
\newcommand{\themename}{\textbf{\textsc{metropolis}}\xspace}

\setbeamertemplate{caption}{\raggedright\insertcaption\par}
\usetikzlibrary{calc,decorations.pathmorphing,patterns}
\pgfdeclaredecoration{penciline}{initial}{
    \state{initial}[width=+\pgfdecoratedinputsegmentremainingdistance,
    auto corner on length=1mm,]{
        \pgfpathcurveto%
        {% From
            \pgfqpoint{\pgfdecoratedinputsegmentremainingdistance}
                      {\pgfdecorationsegmentamplitude}
        }
        {%  Control 1
        \pgfmathrand
        \pgfpointadd{\pgfqpoint{\pgfdecoratedinputsegmentremainingdistance}{0pt}}
                    {\pgfqpoint{-\pgfdecorationsegmentaspect
                     \pgfdecoratedinputsegmentremainingdistance}%
                               {\pgfmathresult\pgfdecorationsegmentamplitude}
                    }
        }
        {%TO
        \pgfpointadd{\pgfpointdecoratedinputsegmentlast}{\pgfpoint{1pt}{1pt}}
        }
    }
    \state{final}{}
}


\title{Physician Behaviors and Hospital Influence}
\date{George Washington University}
\author{Haizhen Lin \& \textbf{Ian McCarthy} \& Michael Richards}
\institute{November 8, 2018}

\begin{document}
\tikzstyle{every picture}+=[remember picture]
\everymath{\displaystyle}

\maketitle

\section{Background}

\begin{frame}{Physician Agency}
    \only<1>{
        Physician with decision-making authority for treatment
        \begin{itemize}
            \item Information asymmetry
            \item Regulatory restrictions
        \end{itemize}
    }
    \only<2->{
        Differential financial incentives between physician and hospital
        \begin{itemize}
            \item More procedures $=$ more revenue, but location of procedure may matter to hospital
            \item Hospital wants less cost with fixed payment, but physician dictates resource use
        \end{itemize}
    }
    \onslide<3->{
        $\longrightarrow$ Incentives for hospitals to influence physicians \\
        \vspace{.25in}
    }
    \onslide<4->{
        \noindent Most direct way (arguably) is to purchase physician practice
    }
\end{frame}

\begin{frame}{Changing Physician Relationships}
    \only<1>{
        \begin{figure}
            \includegraphics[height=2.4in,keepaspectratio]{Richardsetal.png}
            \caption{Richards \textit{et al.}, Medical Care, 2016}
        \end{figure}
    }
    \only<2>{
        \begin{figure}
            \includegraphics[height=2.3in,keepaspectratio]{Bakeretal.jpg}
            \caption{Baker, Bundorf, and Kessler, Health Affairs, 2014}
        \end{figure}
    }
\end{frame}

\begin{frame}{What do we expect from integration?}
    \begin{itemize}
        \item Hospitals claim efficiency gains, reduced fragmentation, increased coordination, etc.
        \item Financial incentives for cost increases and decreases
        \begin{itemize}
            \item[--] Lower costs with fixed payment
            \item[--] Substituting locations of care more efficiently
            \item[--] Spillovers from private insurance
            \item[--] More resources due to pay-for-performance
        \end{itemize}
    \end{itemize}
\end{frame}

\begin{frame}{In context}
    \begin{itemize}
        \item Physician agency (Clemens \& Gottlieb 2014, AER; Afendulis \& Kessler 2007, AER; Gruber \& Owings 1996, RAND; Iizuka 2012, AER)
        \item Vertical integration (Cuellar \& Gertler 2006, JHE; Ciliberto \& Dranove 2006, JHE; Baker \textit{et al.} 2016, JHE; Koch \textit{et al.} 2017, JHE)
    \end{itemize}
\end{frame}

\section{Theoretical Framework}
\begin{frame}{Physician Agency}
    \tikzstyle{na} = [baseline=-.5ex]
    \only<1->{
        Observed care at time $t$ is
        \begin{equation*}
            y_{ijk} = \arg \max_{y}  \theta_{u} \tilde{u} \left(y; \Gamma_{j}, \kappa_{i} \right) + \theta_{\pi} \pi \left(y; \Gamma_{k}, \Gamma_{j} \right).
        \end{equation*} \\
    }
    \only<2->{
        \noindent With linearity and separability assumptions in patient preferences:
        \begin{equation*}
            y_{ijk} =
            \tikz[baseline]{
                \node[fill=blue!20,anchor=base] (t1)
                {$ \alpha_{i} + x_{i}\beta $};
            } +
            \tikz[baseline]{
                \node[fill=red!20, ellipse, anchor=base] (t2)
                {$\Gamma_{jk}$};
            } + \epsilon_{ijk}
        \end{equation*}
        \begin{itemize}[<+-| alert@+>]
            \item[]<3-> Patient Preferences
                \tikz[na] \node[coordinate] (n1) {};
            \item[]<4-> Physician and hospital characteristics
                \tikz[na] \node[coordinate] (n2) {};
        \end{itemize}

        \begin{tikzpicture}[overlay]
            \path[->]<3-> (n1) edge [bend right] (t1);
            \path[->]<4-> (n2) edge [out=0, in=-90] (t2);
        \end{tikzpicture}
    }
\end{frame}

\begin{frame}{Estimation Strategy}
    \tikzstyle{na} = [baseline=-.5ex]
    \only<1-3>{
        Suggests two-step estimation strategy:
        \begin{enumerate}
            \item<2-> Estimate $y_{ijk} = \alpha_{i} + x_{i}\beta + \Gamma_{jk} + \epsilon_{ijk}$ at patient level (separately by year)
            \item<3-> Estimate $\hat{\Gamma}_{jkt} = \gamma_{j} + \gamma_{k} + \tau_{t} + z_{jkt}\delta + \eta_{jkt}$ with physician-hospital panel
        \end{enumerate}
    }
    \only<4>{
        \begin{itemize}
            \item Draws from ``match values'' in labor literature (Abowd \textit{et al.}, 2002; Card \textit{et al.}, 2013, QJE )
            \item Exploits variation across inpatient stays and splits the separation of match value into two steps
            \item Identifies effects on match value from within-physician variation across hospitals (e.g., patient movers in Finkelstein \textit{et al.}, 2016, QJE)
        \end{itemize}
    }
    \only<5-8>{
        Traditional ``match value'' approach:
        \begin{equation*}
            y_{ijk} = \alpha_{i} + x_{i}\beta +
            \tikz[baseline]{
                \node[fill=blue!20,anchor=base] (t1)
                {$\Gamma_{j}$};
            } +
            \tikz[baseline]{
                \node[fill=green!20,anchor=base] (t2)
                {$\Gamma_{k}$};
            } +
            \tikz[baseline]{
                \node[fill=red!20,ellipse,anchor=base] (t3)
                {$\Gamma_{jk}$};
            } +
            \epsilon_{ijk}
        \end{equation*}
        \begin{itemize}[<+-| alert@+>]
            \item[]<6-> Physician effect
                \tikz[na] \node[coordinate] (n1) {};
            \item[]<7-> Hospital effect
                \tikz[na] \node[coordinate] (n2) {};
            \item[]<8-> Physician-hospital match value
                \tikz[na] \node[coordinate] (n3) {};
        \end{itemize}

        \begin{tikzpicture}[overlay]
            \path[->]<6-> (n1) edge [bend right] (t1);
            \path[->]<7-> (n2) edge [out=10, in=-70] (t2);
            \path[->]<8-> (n3) edge [out=0, in=-90] (t3);
        \end{tikzpicture}
    }
    \only<9->{
        Our approach:
        \begin{equation*}
            y_{ijk} = \alpha_{i} + x_{i}\beta + \underbrace{
            \tikz[baseline]{
                \node[fill=blue!20,anchor=base] (t1)
                {$\Gamma_{jk}^{t}$};
            }}_{
            \tikz[baseline]{
                \node[fill=green!20,anchor=base] (t2)
                {$\Gamma_{j}$};
            } +
            \tikz[baseline]{
                \node[fill=red!20,anchor=base] (t3)
                {$\Gamma_{k}$};
            } +
            \tikz[baseline]{
                \node[fill=orange!20,ellipse,anchor=base] (t4)
                {$z_{jkt}\delta$};
            }} +
            \epsilon_{ijk}
        \end{equation*}
        \begin{itemize}[<+-| alert@+>]
            \item[]<10-> Physician, hospital, and match effect (jointly)
                \tikz[na] \node[coordinate] (n1) {};
            \item[]<11-> Physician effect
                \tikz[na] \node[coordinate] (n2) {};
            \item[]<12-> Hospital effect
                \tikz[na] \node[coordinate] (n3) {};
            \item[]<13-> Physician-hospital integration
                \tikz[na] \node[coordinate] (n4) {};
        \end{itemize}

        \begin{tikzpicture}[overlay]
            \path[->]<10-> (n1) edge [out=0, in=-90] (t1);
            \path[->]<11-> (n2) edge [out=10, in=-70] (t2);
            \path[->]<12-> (n3) edge [out=5, in=-80] (t3);
            \path[->]<13-> (n4) edge [out=0, in=-90] (t4);
        \end{tikzpicture}
    }
\end{frame}


\section{Data}
\begin{frame}{Data Sources}
    \begin{itemize}
        \item CMS: 100\% inpatient and institutional outpatient Medicare claims data (2008-2015)
        \item SK\&A: Hospital ownership of physician practices
        \item AHA, HCRIS, POS: Hospital characteristics
        \item Annual IPPS Impact Files: Hospital cost-to-charge ratios (CCR)
        \item ACS: County-level demographics, education, income, and employment
    \end{itemize}
\end{frame}

\begin{frame}{Sample Construction}
    \begin{itemize}
        \item<1-> Planned inpatient stays (elective admissions initiated by a physician, clinic, or HMO referral) and outpatient procedures with observed NPI for the operating physician
        \item<2-> Drop physicians operating in hospitals more than 120 miles from primary office or outside of contiguous U.S.
        \item<3-> Drop physicians with NPIs not matched in the SK\&A data
        \item<4-> Drop lowest/highest 1\% of charges and patients $<$ 65 years old
    \end{itemize}
  \uncover<5->{ $\longrightarrow$ 518,398 unique observations at the physician/hospital/year \\
   $\longrightarrow$ 7.5mm inpatient stays (47\% of total) and 24mm outpatient procedures}
\end{frame}



\section{Estimation of Match Values}
\begin{frame}{Specification}
    \only<1>{
    Two-step estimation strategy:
    \begin{enumerate}
        \item Estimate $y_{ijk} = \alpha_{i} + x_{i}\beta + \Gamma_{jk} + \epsilon_{ijk}$ at patient level (separately by year)
        \item Estimate $\hat{\Gamma}_{jkt} = \gamma_{j} + \gamma_{k} + \tau_{t} + z_{jkt}\delta + \eta_{jkt}$ with physician-hospital panel
    \end{enumerate}
    }
    \only<2>{
    \begin{equation*}
        y_{ijk} = \alpha_{i} + x_{i}\beta + \Gamma_{jk} + \epsilon_{ijk},
    \end{equation*}
    }
\end{frame}

\begin{frame}{Outcomes}
    \begin{equation*}
        \textcolor{red}{y_{ijk}} = \alpha_{i} + x_{i}\beta + \Gamma_{jk} + \epsilon_{ijk},
    \end{equation*}

    \begin{itemize}
        \item Total inpatient and outpatient Medicare payments
        \item Total inpatient and outpatient hospital costs (from cost-to-charge ratios)
        \item Inpatient hospital costs
        \item Inpatient length of stay
        \item Outpatient hospital costs
    \end{itemize}
\end{frame}


\begin{frame}{Independent Variables}
    \begin{equation*}
        y_{ijk} = \textcolor{red}{\alpha_{i}} + x_{i}\beta + \Gamma_{jk} + \epsilon_{ijk},
    \end{equation*}

    \begin{itemize}
        \item Quartiles of total ``other'' Medicare payments and procedures
        \item Covers 2008 through 2015 period
        \item Beneficiary-specific measure of ``utilization''
    \end{itemize}
\end{frame}

\begin{frame}{Independent Variables}
    \begin{equation*}
        y_{ijk} = \alpha_{i} + \textcolor{red}{x_{i}}\beta + \Gamma_{jk} + \epsilon_{ijk},
    \end{equation*}

    \begin{itemize}
        \item Age, gender, race
        \item Indicators for ICD9 diagnosis code groups (18 diagnosis groups per variable plus missing group)
        \item Indicators for primary DRGs (with at least 1000 observations in a given year)
        \item Minor differences between total, inpatient, and outpatient specifications
    \end{itemize}
\end{frame}

\begin{frame}{Summary of Match Values}
    \only<1>{
        \metroset{block=fill}
        \begin{block}{1. Calculate Cost Differential}
            Apply minimum cost physician-hospital combination to all of physician $j$'s patients:
            \begin{align*}
                \Delta_{k} y_{ij} &= \hat{y}_{ijk} - \hat{y}_{ij\underbar{k}} \nonumber \\
                   &= \hat{\alpha}_{i} + x_{i}\hat{\beta} + \hat{\Gamma}_{jk} - \hat{\alpha}_{i} - x_{i}\hat{\beta} - \min \left\{\Gamma_{j1},...,\Gamma_{jK} \right\} \nonumber \\
                   &= \hat{\Gamma}_{jk} - \min \left\{\Gamma_{j1},...,\Gamma_{jK} \right\}.
            \end{align*}
        \end{block}
    }
    \only<2>{
        \metroset{block=fill}
        \begin{block}{2. Summarize}
            \begin{itemize}
                \item Total cost differential for each physician
                \item Limit to pairs with 5 or more procedures
                \item Limit to physicians with 2 or more hospitals in a year
                \item Present interquartile range and mean
            \end{itemize}
        \end{block}
    }

\end{frame}

\begin{frame}{Within-physician Variation in Payments}
    \only<1>{
        \begin{figure}
            \centering
            \includegraphics[height=2.5in,width=5in,keepaspectratio]{PhyPay_All_Graph}
        \end{figure}
    }
    \only<2>{
        \begin{figure}
            \centering
            \includegraphics[height=2.5in,width=5in,keepaspectratio]{PhySave_All_Graph}
        \end{figure}
    }
\end{frame}

\begin{frame}{Within-hospital Variation in Costs}
    \begin{figure}
        \centering
        \includegraphics[height=2.5in,width=5in,keepaspectratio]{HospSave_Graph_DRG}
    \end{figure}
\end{frame}


\section{Estimation of Hospital Influence}
\begin{frame}{Specification}
    \only<1>{
    Two-step estimation strategy:
    \begin{enumerate}
        \item Estimate $y_{ijk} = \alpha_{i} + x_{i}\beta + \Gamma_{jk} + \epsilon_{ijk}$ at patient level (separately by year)
        \item Estimate $\hat{\Gamma}_{jkt} = \gamma_{j} + \gamma_{k} + \tau_{t} + z_{jkt}\delta + \eta_{jkt}$ with physician-hospital panel
    \end{enumerate}
    }
    \only<2>{
    \begin{equation*}
        \hat{\Gamma}_{jkt} = \gamma_{j} + \gamma_{k} + \tau_{t} + z_{jkt}\delta + \eta_{jkt},
    \end{equation*}
    }
\end{frame}

\begin{frame}{Main Outcomes}
    \begin{equation*}
        \textcolor{red}{\hat{\Gamma}_{jkt}} = \gamma_{j} + \gamma_{k} + \tau_{t} + z_{jkt}\delta + \eta_{jkt},
    \end{equation*}

    \begin{table}[htb!]
    \centering
    \footnotesize
    \centerline{
    \begin{tabular}{l|rrrrr|r}
        & 2008  & 2012 & 2013 & 2014 & 2015 & Overall \\
        \hline
Total Payments &      7,152         &      8,171         &      8,501         &      8,941         &      9,169         &      8,094         \\
            &    (7,595)         &    (8,472)         &    (8,290)         &    (8,724)         &    (8,755)         &    (8,228)         \onslide<2->{\\
Total Costs &      9,387         &     11,323         &     11,756         &     12,237         &     12,736         &     10,965         \\
            &    (9,632)         &   (10,954)         &   (10,906)         &   (11,549)         &   (11,728)         &   (10,626)         \onslide<3->{\\
Inpatient Costs &     13,655         &     16,958         &     17,711         &     18,367         &     19,081         &     16,294         \\
            &    (7,752)         &    (9,407)         &    (9,612)         &    (9,997)         &   (10,184)         &    (9,256)         \onslide<4->{\\
Inpatient LOS &       5.984         &       6.021         &       6.002         &       6.062         &       6.029         &       5.960         \\
            &     (2.427)         &     (2.493)         &     (2.494)         &     (2.513)         &     (2.492)         &     (2.436)         \onslide<5->{\\
Outpatient Costs &      3,007         &      3,806         &      4,014         &      4,190         &      4,361         &      3,693         \\
            &    (2,135)         &    (2,782)         &    (2,925)         &    (3,096)         &    (3,195)         &    (2,749)         }}}}\\
    \end{tabular}}
    \end{table}

\end{frame}


\begin{frame}{Independent Variables}
    \begin{equation*}
        \hat{\Gamma}_{jkt} = \gamma_{j} + \gamma_{k} + \tau_{t} + \textcolor{red}{z_{jkt}}\delta + \eta_{jkt},
    \end{equation*}

    \begin{table}[htb!]
    \centering
    \footnotesize
    \centerline{
    \begin{tabular}{l|rrrrr|r}
        & 2008  & 2012 & 2013 & 2014 & 2015 & Overall \\
        \hline
Integrated       &       0.130         &       0.206         &       0.233         &       0.255         &       0.332         &       0.196         \\
                 &     (0.336)         &     (0.404)         &     (0.422)         &     (0.436)         &     (0.471)         &     (0.397)         \onslide<2->{\\
Physician FTE       &       24.23         &       28.59         &       31.14         &       31.74         &       33.13         &       28.43         \\
                    &     (99.28)         &     (109.8)         &     (120.5)         &     (120.0)         &     (119.5)         &     (110.9)         \onslide<3->{\\
Resident FTE        &       25.77         &       28.45         &       29.13         &       30.69         &       30.97         &       28.08         \\
                    &     (108.2)         &     (120.4)         &     (121.4)         &     (125.9)         &     (127.8)         &     (117.8)         \onslide<4->{\\
Nurse FTE           &       340.8         &       365.7         &       369.1         &       384.9         &       402.7         &       364.8         \\
                    &     (446.8)         &     (487.8)         &     (494.8)         &     (519.1)         &     (550.7)         &     (487.3)         \onslide<5->{\\
Other FTE           &       749.9         &       763.0         &       761.8         &       776.4         &       806.0         &       762.8         \\
                    &     (975.5)         &    (1032.4)         &    (1076.2)         &    (1101.5)         &    (1157.2)         &    (1037.4)         \onslide<6->{\\
Beds (100s)         &       1.980         &       1.967         &       1.958         &       1.982         &       2.009         &       1.976         \\
                    &     (2.160)         &     (2.142)         &     (2.137)         &     (2.172)         &     (2.235)         &     (2.154)         }}}}}\\
    \end{tabular}}
    \end{table}

\end{frame}

\begin{frame}{Independent Variables}
    \begin{equation*}
        \hat{\Gamma}_{jkt} = \gamma_{j} + \gamma_{k} + \tau_{t} + \textcolor{red}{z_{jkt}}\delta + \eta_{jkt},
    \end{equation*}

    \begin{table}[htb!]
    \centering
    \footnotesize
    \centerline{
    \begin{tabular}{l|rrrrr|r}
        & 2008  & 2012 & 2013 & 2014 & 2015 & Overall \\
        \hline
Practice Size&       13.73         &       17.31         &       17.31         &       17.82         &       18.41         &       16.10         \\
             &     (32.10)         &     (30.70)         &     (29.28)         &     (28.46)         &     (28.02)         &     (30.05)         \onslide<2->{\\
Experience   &       22.55         &       23.00         &       23.94         &       23.65         &       24.77         &       23.17         \\
             &     (6.496)         &     (6.703)         &     (6.950)         &     (6.902)         &     (6.989)         &     (6.746)         \onslide<3->{\\
\% Multi-Specialty &       0.249         &       0.248         &       0.266         &       0.284         &       0.344         &       0.264         \\
\% with Surgery &       0.452         &       0.501         &       0.507         &       0.508         &       0.454         &       0.480         }}\\
    \end{tabular}}
    \end{table}

\end{frame}

\begin{frame}{Estimated Effects of Vertical Integration}
    \vspace{0.75in}
    \begin{table}[htb!]
    \centering
    \centerline{
    \begin{tabular}{l|rr}
        Outcome & Estimate & St. Error \\
        \hline\hline \onslide<2->{\vspace{-.1in}\\
        Total Medicare Payments & 110.945** & (46.768)  \onslide<3->{\\
        Total Hospital Costs & 255.126***  & (64.621) \onslide<4->{\\
        Inpatient Hospital Costs & 209.579*** & (53.671) \onslide<5->{\\
        Inpatient Length of Stay & -0.028 & (0.019) \onslide<6->{\\
        Outpatient Hospital Costs & -58.581*** & (20.320) }}}}}\\
        \hline
        \multicolumn{3}{l}{* p-value $<$0.1, ** p-value $<$0.05, *** p-value $<$0.01}
    \end{tabular}}
    \end{table}
\end{frame}

\begin{frame}{Threats to Identification and Interpretation}
    Estimator is effectively a two-way fixed effects DD with time varying treatment
    \only<2>{
        \metroset{block=fill}
        \begin{block}{Potential Problems}
            \begin{enumerate}
                \item Vertical integration due to time-varying unobservables \& outcomes (standard DD concern)
                \item Weighted average of all 2$\times$2 DD estimates, with some potentially negative weights
            \end{enumerate}
        \end{block}
    }
\end{frame}

\begin{frame}{Event Study: Total Medicare Payments}
    \begin{figure}
        \centering
        \includegraphics[height=2.5in,width=5in,keepaspectratio]{EventPay_All_2011}
    \end{figure}
\end{frame}

\begin{frame}{Event Study: Total Hospital Costs}
    \begin{figure}
        \centering
        \includegraphics[height=2.5in,width=5in,keepaspectratio]{EventCharge_All_2011}
    \end{figure}
\end{frame}

\begin{frame}{Event Study: Inpatient Hospital Costs}
    \begin{figure}
        \centering
        \includegraphics[height=2.5in,width=5in,keepaspectratio]{EventCharge_IP_2011}
    \end{figure}
\end{frame}


\begin{frame}{Takeaways}
    \begin{itemize}
        \item Evidence of increase in payments and costs
        \item Evidence consistent with common trends assumption for total payments and costs
        \item Some concern about common trends for inpatient costs
    \end{itemize}
\end{frame}


\begin{frame}{Endogeneity of physician-hospital integration}
    \label{ivs}
    \only<1>{
        Integration could be driven by:
        \begin{itemize}
            \item Unobserved, time-varying practice characteristics
            \item Existing costs and treatment patterns
        \end{itemize}
    }
    \only<2>{
        \metroset{block=fill}
        \begin{block}{1. Set of possible physician-hospital pairs}
            Form set of all hospitals where physician operates from 2008-2015
        \end{block}
    }
    \only<3>{
        \metroset{block=fill}
        \begin{block}{2. Estimate probability of integration}
            \begin{equation*}
                \text{Pr}\left(I_{jk}=1\right) = \frac{\text{exp}\left(\lambda z_{jk}\right)}{1+\text{exp}\left(\lambda z_{jk}\right)}
            \end{equation*}
            \vspace{-.3in}
            \begin{itemize}
                \item Hospital and practice characteristics
                \item Average differential distance (relative to nearest hospital in patient choice set)
                \item Differential distance interacted with hospital and practice characteristics
            \end{itemize}
        \end{block}
    }
    \only<4>{
        \metroset{block=fill}
        \begin{block}{2. Estimate probability of integration}
            \begin{equation*}
                \hat{\text{Pr}}\left(I_{jk}=1\right) = \frac{\text{exp}\left(\hat{\lambda} z_{jk}\right)}{1+\text{exp}\left(\hat{\lambda} z_{jk}\right)}
            \end{equation*}
            \noindent Intuition: Physicians less likely to seek/allow acquisition if patients live further away
        \end{block}
    }
    \only<5>{
        \metroset{block=fill}
        \begin{block}{2. Estimate probability of integration}
            \begin{equation*}
                \hat{\text{Pr}}\left(I_{jk}=1\right) = \frac{\text{exp}\left(\hat{\lambda} z_{jk}\right)}{1+\text{exp}\left(\hat{\lambda} z_{jk}\right)}
            \end{equation*}
            \noindent Intuition: Physicians less likely to seek/allow acquisition if patients live further away
        \end{block}
        \begin{equation*}
            \hat{\Gamma}_{jkt} = \gamma_{j} + \gamma_{k} + \tau_{t} + \underbrace{I_{jkt}}_{\mathclap{\hat{I}_{jkt}=\hat{\text{Pr}}(I_{jkt}=1)}} \delta_{1}  + \tilde{z}_{jkt}\delta_{2} + \eta_{jkt},
        \end{equation*}
    }
    \only<6->{
    \vspace{0.75in}
    \begin{table}[htb!]
    \centering
    \centerline{
    \begin{tabular}{l|rr}
        Outcome & Estimate & St. Error \\
        \hline\hline \onslide<7->{\vspace{-.1in}\\
        Total Medicare Payments & 1032.112** & (498.814)  \onslide<8->{\\
        Total Hospital Costs & 3213.162***  & (696.032) \onslide<9->{\\
        Inpatient Hospital Costs & 3081.788*** & (533.495) \onslide<10->{\\
        Inpatient Length of Stay &  0.108 & (0.179) \onslide<11->{\\
        Outpatient Hospital Costs & -337.977* & (204.733) }}}}}\\
        \hline
        \multicolumn{3}{l}{* p-value $<$0.1, ** p-value $<$0.05, *** p-value $<$0.01}
    \end{tabular}}
    \end{table}
    }
\end{frame}


\section{Allocation of Procedures and Patients}
\begin{frame}{Other Effects}
    Other ways integration posited to affect physician behavior:
    \begin{itemize}
        \item More procedures overall (not per patient)
        \item Reallocating procedures from other hospitals
        \item Reallocating procedures across inpatient and outpatient settings
    \end{itemize}
\end{frame}

\begin{frame}{Results on Other Outcomes}
    \vspace{.75in}
    \begin{table}[htb!]
    \centering
    \centerline{
    \begin{tabular}{l|rr}
        Outcome & Estimate & St. Error \\
        \hline\hline \onslide<2->{\vspace{-.1in}\\
        Physician's inpatient share & 0.065*** & (0.003) \onslide<3->{\\
        Physician's outpatient share & 0.047***  & (0.003) \onslide<4->{\\
        Total patients & 6.892*** & (0.527) \onslide<5->{\\
        Inpatient procedures & 0.784*** & (0.169) \onslide<6->{\\
        Outpatient procedures & 9.929*** & (1.087) }}}}}\\
        \hline
        \multicolumn{3}{l}{* p-value $<$0.1, ** p-value $<$0.05, *** p-value $<$0.01}
    \end{tabular}}
    \end{table}
\end{frame}

\begin{frame}{Summary of Results}
    \only<1>{
        \metroset{block=fill}
        \begin{block}{Effects per Patient}
            \begin{itemize}
                \item Increase in Medicare payments (\$110 to \$300) and hospital costs (\$255-\$500)
                \item Extrapolates to between \$77 and \$210 million in added Medicare payments from vertical integration
            \end{itemize}
        \end{block}
    }

    \only<2>{
        \metroset{block=fill}
        \begin{block}{Sensitivity}
           \begin{itemize}
                \item Event study consistent with common trends for Medicare payments and total hospital costs
                \item Calculation of 2$\times$2 DD weights suggests relatively small portion of negative weights (70\% positive weights)
                \item As falsification test, no effects on payments or DRG weights per inpatient stay
            \end{itemize}
        \end{block}
    }

    \only<3>{
        \metroset{block=fill}
        \begin{block}{Effects on Total Patients and Allocation of Procedures}
           \begin{itemize}
                \item More procedures going to acquiring hospital
                \item New procedures predominantly coming from outpatient side (13 new outpatient procedures per inpatient procedure)
            \end{itemize}
        \end{block}
    }

    \only<4>{
        \metroset{block=fill}
        \begin{block}{Interpreting Main Results}
           \begin{itemize}
                \item Total within-physician variation in Medicare payments of around \$140,000 per physician per year
                \item Increases due to vertical integration of between \$110 and \$300 per patient per year
                \item 5-13\% of within-physician variation explained by vertical integration
            \end{itemize}
        \end{block}
    }

\end{frame}

\section*{Thank You}


\end{document}






